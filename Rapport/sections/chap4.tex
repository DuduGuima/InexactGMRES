
(Not results in the cluster, left to remake them after finishing the discussions we had about residuals)
(I thought about changing the beggining of this chapter as a whole small chapter about BEM)

(Also, I thought about using the graphs with the bounds of each expression in the article, but I'm not shure if we put them here or with the Inexact theory chapter above)

Before using the inexact product in more complex problems, simpler examples are used to validate the approach and fix minor parameters in the scheme.

The two firsts tests evaluate the speedup in the product of a Hierarchical Matrix and a vector and an exection of the Inexact GMRES with few iterations, using the operators obtained through 2nd type Equations of Laplace and Helmholtz \ref{eq:pde_helmlaplace}, where the last one is a scaterring problem, where $\Delta$ is the Laplace operator and everything is suposed to be solved in two dimentions. The last test will use a cavity problem to test the speedup of the algorithm in a situation with more iterations.

\begin{align}\label{eq:pde_helmlaplace}
    \begin{split}
        \Delta u &= 0 \\
        \Delta u + k^{2}u &= 0
    \end{split}
\end{align}

Reformulationg both equations as a Boundary Integral Equation, the simple \textit{direct} formulation is used to write the solution as \ref{eq:direct_formulation}, where $\Gamma$ is boudnary of the domain, \textit{S} and \textit{D} are the single and double layer operators, defined as \ref{eq:single_double}, and $G(x,y)$ is the fundamental solution of the desired PDE.

\begin{equation}\label{eq:direct_formulation}
    -\frac{u(x)}{2} + D[u](x) = S[\partial_{\nu }u] (x), \hspace{0.3in} x \in \Gamma
\end{equation}

\begin{align}\label{eq:single_double}
    \begin{split}
        S[\sigma](x) &= \int_{\Gamma} G(x,y) \sigma(y)  \,ds(y) \\
        D[\sigma](x) &= \int_{\Gamma} \frac{\partial G}{\partial \nu_{y}}(x,y) \sigma(y)  \,ds(y)
    \end{split}
\end{align}

For the first to examples, a unit circle around the origin is used as the boundary to generate the operators, with the mesh being created with the Inti library \cite{git-inti}.

For the last test, the mesh is made from a cavity $.geo$ file avaiable in \cite{git_dudu}. A view of the figure can be seen in \ref{fig:cavity_fig}. The incident wave' angle is chosen to be $\frac{\pi}{4}$ rad.

\begin{figure}[h!]
    \centering
    \includegraphics[width=0.6\linewidth]{images/cavity_fig.jpg}
    \caption{Geometry used in the test.}
    \label{fig:cavity_fig}
\end{figure}

A good way to infer the maximum acceleration possible for the inexact products would be using only admissible rank 1 blocks and measuring its execution time. Although such thing would not happen in a practical situation, it gives a maximum bound for the speed up we should expect.
For doing that, the product tolerance is changed to \textit{Infinity}, and the product will be realised with only rank-1 blocks, since it's programmed to get the first aproximation lower than its given tolerance.

\subsection{Laplace's results}

Setting our product tolerance to \textit{Infinity} and using only rank-1 blocks in the product, we got a () speedup.

All results are contained in \ref{fig:laplace_results}, showing the evolution of the residual with the product tolerance aswell as the speedup to each of these values.

\begin{figure}[h!]
    \centering
    \includesvg[width=0.6\linewidth]{images/fixed_size_laplace_product.svg}
    \caption{Speedup and residual evolution for the product between a 8000x8000 HMatrix and a vector.}
    \label{fig:laplace_results}
\end{figure}

\subsection{Helmholtz's results}

For a maximum speedup bound in the product, the infinity tolerance brought a () speedup.

For the unitary circle boundary the results can be seen in \ref{fig:Helmholtz_circle_results}.

\begin{figure}[h!]
    \centering
    \begin{subfigure}[b]{0.45\linewidth}
        \includesvg[width=\linewidth]{images/fixed_size_helmholtz_product.svg}
        \caption{Results for the product of a 70000x 70000 HMatrix and a vector.}
    \end{subfigure}
    \begin{subfigure}[b]{0.45\linewidth}
        \includesvg[width=\linewidth]{images/fixed_size_helmholtz_gmres.svg}
        \caption{Results for an initial application of the Inexact GMRES algorithm.}
    \end{subfigure}
    \caption{Results for the application of the Inexact GMRES algorithm with a 70000x70000 HMatrix.}
    \label{fig:Helmholtz_circle_results}
\end{figure}


For the cavity, \ref{fig:cavity_results}.

\begin{figure}[h!]
    \centering
    \includesvg[width=0.6\linewidth]{images/fixed_size_helmholtz_cavity.svg}
    \caption{Speedup witnessed in the application of the Inexact GMRES in a 50000x50000 matrix.}
    \label{fig:cavity_results}
\end{figure}

An evolution of the number of iterations in face of the different tolerances passed to the algorithm is in \ref{fig:cavity_iterations}.

\begin{figure}[h!]
    \centering
    \includesvg[width=0.6\linewidth]{images/fixed_size_helmholtz_cavity_iterations.svg}
    \caption{Evolution of the quantity of iterations needed for convergence and overall tolerance passed as an argument.}
    \label{fig:cavity_iterations}
\end{figure}


%Fazer analise dos dois extremos

To start assessing the maximum gain possible, we start by initiating the product tolerance as infitine and seeing the result. Choosing an infinite tolerance grants us the all admissible block used in the products will have rank 1.


