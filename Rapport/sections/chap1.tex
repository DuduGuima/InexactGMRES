\subsection{Iterative Methods and motivation}

%Iterative methods appear as an alternative to exact solution methods, where the true solution is not desired and a good approximation is enough.

%Iterative methods appear as an alternative to direct solution methods, where the true/exact solution to a problem is only desired up to a certaion tolerance, usually used when the exact methods need heavy computing power and/or are inefficient in the current context.
Iterative methods appear as an alternative to direct solution methods, where the direct solution to a problem usually scales with $\mathcal{O}(m^{3})$ complexity, where $m$ is the dimension of the input matrix. Since larger matrices are usually employed in practice, the direct algorithms become inefficient, and we require a more reliable approach.


The idea of the iterable methods is to find, after a certain number of iterations, a sequence ${x_{k}}$ that converges to $x$, the exact solution of the problem \ref{eq:suite}. This should be done while making the large-scale computations faster, i.e., obtaining a complexity smaller than $\mathcal{O}(m^{3})$, and keeping a maximum tolerance between the iterable solution and the exact one.


\begin{equation}\label{eq:suite}
    x = \lim_{k \to \infty} x_{k}
\end{equation}


The method stops after $k$ iterations, where $x_{k}$ is the fist element of the sequence to satisfy the condition \ref{eq:residual}.

\begin{equation}\label{eq:residual}
    \frac{||Ax_{k} - b||}{||b||} \leq \epsilon
\end{equation}

We define $\epsilon$ as the tolerance given to the algorithm.

To achieve a smaller complexity than $\mathcal{O}(m^{3})$, Iterative Methods employ matrix vector products, complexity $\mathcal{O}(m^{})$, instead of the product between matrices found in direct methods. So, considering an Iterative Method finds a solution in $k$ steps, its complexity would be $\mathcal{O}(km^{2})$.

Therefore, guaranteeing that the convergence rate of the method is sufficiently fast gives a $k << m$, and the Interative Method can be way more efficient thatn its counterpart.

The method we employ in our problems is the GMRES, explained later in the report. Its main ideia involves the projection of a high dimensional problem, as large as $A$ in \ref{eq:residual}, in a lower dimensional \textit{Krylov Space}:

\begin{equation}\label{eq:xkry}
    x_{k} \in \spn(b,Ab,A^{2}b,...,A^{k-1}b)
\end{equation}

Explained in more detail below.

\subsection{Krylov's Subspace}
Be $A \in \mathbb{K}^{n \times n}$ a matrix and $b\in \mathbb{K}^{n}$. To each $k\leq n$ the Krylov's Subspace $\mathcal{K}_{k}=\mathcal{K}_{k}(A,b)$ associated to A, b is defined as \ref{eq:krylov}.

\begin{equation}\label{eq:krylov}
    \mathcal{K}_{k}(A,b) = \spn(b,Ab,A^{2}b,\dots , A^{k-1}b)
\end{equation}

These Subspaces also have the following property: $k<l \to \mathcal{K}^{k} \subset \mathcal{K}^{l}$ \cite{bonnet}.

The subspace $\mathcal{K}_{k}(A,b)$ is also the subspace of all the vectors from $\mathbb{R}^{m}$ which could be written as $x=p(A)b$, where $p(A)$ is a polynom of degree less than $k-1$ which $p(0)=1$.

The problem with using ${A^{k}b}, k \in {0,1,2,\dots}$ as a base comes from the fact that successive products of $A$ make vectors that are \textit{approximately colinear}, since those are really close of the eigenvector with the largest eigenvalue of $A$.

\subsection{Arnoldi's Method}


Arnoldi's Method is an orthogonal projection method used to find an orthonormal basis ${q_{1}, \dots, q_{k}}$ to $\mathcal{K}_{k}(A,b)$. An algorithm for the method can be found in \ref{alg:arnoldi}.


\begin{algorithm}
    \caption{Arnoldi's iteration}\label{alg:arnoldi}
    \begin{algorithmic}[1]
        \State $A \in \mathbb{K}^{n \times n}$ et $b\in \mathbb{K}^{n}$
        \State $x=0, \beta=\norm{b},q_{1}=\frac{b}{\beta}$

        \For{$j=1,2,\dots k$}
        \State $q_{j+1} = Aq_{j}$

        \For{ $i=1,2,\dots j$}
        \State $h_{ij}= q_{j+1}^{t}q_{i}$
        \State $q_{j+1} = q_{j+1} - h_{ij}q_{i}$
        \EndFor
        \State $h_{j+1,j}=\norm{q_{j+1}}$
        \State $q_{j+1} = \frac{q_{j+1}}{h_{j+1,j}}$
        \EndFor

    \end{algorithmic}
\end{algorithm}

As we can see, at each step in \ref{alg:arnoldi}, the previous vector $q_{j}$ is multiplied by $A$ and then orthonormalized in relation to all previous $q_{i}$'s with a Gram-Schmidt procedure. If $q_{j+1}$ ever vanishes during the inner loop between lines 5 and 8, the algorithm stops.

What is left is to show the $q_{i}$ generated by \ref{alg:arnoldi} form an othornormal basis for $\mathcal{K}_{k}(A,b)$.

\begin{proof}\label{proof:arnoldi}
    By construction $q_{j}$, $j = 1,2,\dots, k$ are othornormal. To show they span $\mathcal{K}_{k}(A,b)$ we prove $q_{j}$ has the form $p_{j-1}(A)b$, where $p_{j}(A)$ is a polynomial of degree $j-1$ in A.
    Using induction the result is true for $j=1$ since $q_{1} = b$. We assume the result is true for all integers $\leq j$ and consider $q_{j+1}$. Using the definition of $q_{j+1}$ in \ref{alg:arnoldi} we have:

    \begin{equation}
        h_{j+1, j}q_{j+1} = Aq_{j} - \sum_{i=1}^{j} h_{ij}q_{i} = Ap_{j-1}(A)b - \sum_{i=1}^{j} h_{ij}p_{i-1}(A)b
    \end{equation}

    Since, by the induction step above, $q_{i} = p_{i-1}(A)b$.

    This shows $q_{j+1}$ can be written as $p_{j}(A)b$ and completes the proof.
\end{proof}

We also make note of the fact $q_{1} = \frac{b}{||b||}$.

If we denote by $Q$ the $n x k$ matrix with column vectors $q_{1}, \dots, q_{k}$ found in \ref{alg:arnoldi} and $H_{k}$ the $(k + 1)x k$ Hessenberg matrix whose nonzero entries $h_{ij}$ are given just as in \ref{alg:arnoldi}, we have \ref{eq:init_arnoldi}.

\begin{equation} \label{eq:init_arnoldi}
    AQ_{k} = Q_{k+1}H_{k}
\end{equation}

\begin{proof}
    For each column-vector of $Q$, $q_{i}$, \ref{eq:init_arnoldi} could be written as \ref{eq:final_arnoldi}, where the representation of $\mathcal{K}_{k}(A,b)$ with an orthonormal basis becomes more evident.

    \begin{equation}\label{eq:final_arnoldi}
        Aq_{m} = h_{1m}q_{1} + h_{2m}q_{2} + \dots h_{m+1,m}q_{m+1}
    \end{equation}

    This relation can be directly seen in \ref{alg:arnoldi} by using line 10 and the inner loop between lines 5 and 8:

    \begin{align}
        \begin{split}
            q_{m+1}h_{m+1,m} & = Aq_{m} - \sum_{i=1}^{m} h_{im}q_{i}\\
            Aq_{m} & = \sum_{i=1}^{m+1} h_{im}q_{i}
        \end{split}
    \end{align}

\end{proof}


