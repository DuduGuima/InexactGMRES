
A projection in $\mathcal{K}_{k}(A,b)$, where we take the different approximations as in \ref{eq:init_gmres}, where $Q_{m}$ is the vector in \ref{eq:init_arnoldi}.
\begin{equation}\label{eq:init_gmres}
    x = x_{0} + Q_{m}y
\end{equation}

With \ref{eq:init_gmres} and \ref{eq:init_arnoldi} the residual becomes \ref{eq:final_gmres}, where $x_{0} = 0$, $\beta=\norm{b}$ and $Q_{m+1}^{t}b=(\norm{b} 0 \hspace{0.05in} 0\dots)^{t}$ since the columns of $Q_{m+1}$ are orthonormal vectors and $q_{1} = \frac{b}{\norm{b}}$.

\begin{align} \label{eq:final_gmres}
    \begin{split}
        r(y) &= \norm{b - Ax}\\
        &= \norm{b - A(Q_{m}y)}\\
        &= \norm{b-Q_{m+1}H_{m}y} \\
        &= \norm{Q_{m+1}(Q_{m+1}^{t}b-H_{m}y)} \\
        &= \norm{\beta e_{1} - H_{m}y}
    \end{split}
\end{align}


Thus, $y$ which appears in \ref{eq:init_gmres}, is found as the solution of the residual's minimisation problem  in \ref{eq:final_gmres}.

\begin{equation}\label{eq:y_gmres}
    y = \min_{y} \norm{\beta e_{1} - H_{m}y}
\end{equation}

An initial version of the GMRES is in \ref{alg:gmres_init}. The lines 4 to 12 contain the Arnoldi's Method presented in \ref{alg:arnoldi}.

\begin{algorithm}
    \caption{Initial GMRES}\label{alg:gmres_init}
    \begin{algorithmic}[1]
        \State $A \in \mathbb{K}^{n \times n}$ and $b\in \mathbb{K}^{n}$
        \State $x=0, \beta=\norm{b},q_{1}=\frac{b}{\beta}$
        \For{$k=1,2,\dots$}
        \For{$j=1,2,\dots k$}
        \State $q_{j+1} = Aq_{j}$

        \For{ $i=1,2,\dots j$}
        \State $h_{ij}= q_{j+1}^{t}q_{i}$
        \State $q_{j+1} = q_{j+1} - h_{ij}q_{i}$
        \EndFor
        \State $h_{j+1,j}=\norm{q_{j+1}}$
        \State $q_{j+1} = \frac{q_{j+1}}{h_{j+1,j}}$
        \EndFor
        \State Find $y = \min_{y} \norm{\beta e_{1} - H_{m}y}$
        \State $x = Q_{k}y$
        \State \textbf{Stop} if the residual is smaller than the tolerance
        \EndFor
    \end{algorithmic}
\end{algorithm}

However, \ref{alg:gmres_init} doesn't present an efficient way of finding the residual in each iteration. To solve this problem and also to find a more efficient way of solving the least squares in \ref{eq:y_gmres}, we apply a transformation to $H_{m}$, turning it into a triangular matrix.

\subsection{Givens's Rotation}

Givens's operator, $G(i,i+1)$, is a unitary matrix such that the column vector $a = Gb$ has the elements $a(i) = r \in \mathbb{R}$ and $a(i+1)=0$. It has a structure as in \ref{eq:givens}. The coefficients $c_{i},s_{i}$ only appear in the rows $i$ et $i+1$.

\begin{equation}\label{eq:givens}
    G(i,i+1)=
    \begin{bmatrix}
        1 &        &   &        &       &   &        &   \\
          & \ddots &   &        &       &   &        &   \\
          &        & 1 &        &       &   &        &   \\
          &        &   & c_{i}  & s_{i} &   &        &   \\
          &        &   & -s_{i} & c_{i} &   &        &   \\
          &        &   &        &       & 1 &        &   \\
          &        &   &        &       &   & \ddots &   \\
          &        &   &        &       &   &        & 1 \\
    \end{bmatrix}
\end{equation}
This operator offers a way to transform the columns in $H_{m}$, \textit{zeroing} the elements outside the main diagonal. Since a product of unitary operators is still unitary, \ref{eq:y_gmres} can be written as \ref{eq:after_givens}, where $R_{m}$ and $g_{m}$ are the results from the application of multiple Givens's operators to $H_{m}$ and $\beta e_{1}$.

\begin{equation}\label{eq:after_givens}
    y = \min_{y} \norm{\beta e_{1} - H_{m}y} = \min_{y} \norm{g_{m} - R_{m}y}
\end{equation}


Thus, the new problem \ref{eq:after_givens} can be solved with a simple backwards substitution. If $g_{m} = [\gamma_{1} \dots \gamma_{m+1}]^{t}$, an $m+1$ column vector, and $\{ R_{m} \}_{ij} = r_{ij}$ an $m+1$ by $m$ upper triangular matrix with $r_{ii} \neq 0$ and its last row filed with zeros, each element of $y_{m} = [y_{1} \dots y_{m}]$ is given by \ref{eq:triangular_system}.

\begin{align}\label{eq:triangular_system}
    \begin{split}
        \gamma_{k} &= \sum_{i=k}^{m} r_{ki} y_{i}\\
        y_{m} &= \frac{\gamma_{m}}{r_{mm}} \\
        y_{i} &= \frac{1}{r_{ii}} \left( \gamma_{i} - \sum_{j=i+1}^{m} r_{ij} \gamma_{j}  \right)
    \end{split}
\end{align}

A simple algorithm to this end can be written as \ref{alg:triang_system}.

\begin{algorithm}
    \caption{Backwards substitution}\label{alg:triang_system}
    \begin{algorithmic}[1]
        \State $A \in \mathbb{K}^{n \times n}, \{ A \}_{ij} = a_{ij}$ and $b\in \mathbb{K}^{n}$
        \For{$k=n,n-1,\dots$}
        \State $y_{k} = b_{k}$
        \For{$j=n,n-1,\dots k+1$}
        \State $y_{k} = y_{k} - a_{kj}y_{j}$
        \EndFor
        \State $y_{k} = \frac{y_{k}}{a_{kk}}$
        \EndFor
    \end{algorithmic}
\end{algorithm}



It can be shown that $g_{m}$ also contains the residual of each iteration \cite{saad2003iterative}. Since it's an $m+1$ column vector, we have, with $\Omega_{m}$ being the necessary Givens's Rotations to make $H_{m}$ upper triangular \ref{eq:proof_residual}.

\begin{align}\label{eq:proof_residual}
    \begin{split}
        \norm{b - Ax_{m}} & = \norm{Q_{m+1}^{t}(\beta e_{1} - H_{m}y_{m})}\\
        \norm{\beta e_{1} - H_{m}y_{m}} &= \norm{\Omega_{m}^{t} (g_{m} - R_{m}y_{m})}
    \end{split}
\end{align}

Since $y_{m}$ is a solution to the system, the norm in \ref{eq:proof_residual} is defined by $\norm{\gamma_{m+1} - (R_{m})_{m+1,1:m} y_{m}}$.

But since the last row in $R_{m}$ is composed by zeros, we have that $\norm{b - Ax_{m}} = |\gamma_{m+1}|$, which gives a more efficient way to obtain the residuals during each iteration.


\subsection{Inexact GMRES}

The heaviest part in the code is in the matrix-vector product \ref{alg:arnoldi}, line 4. Therefore, one approach to accelerate the iterations involves an approximation of $Aq $, instead of using the exact answer, as shown in \ref{eq:aprox_Aq}.

\begin{equation}\label{eq:aprox_Aq}
    \mathcal{A}q = (A + E)q
\end{equation}

Where \textit{E} in \ref{eq:aprox_Aq} is a \textit{pertubation matrix} that changes with each iteration and will be written as $E_{k}$ for iteration k.

When we realise the inexact matrix-vector product, instead of the regular one, the left side of \ref{eq:init_arnoldi} must be changed by \ref{eq:new_projection}.


\begin{align} \label{eq:new_projection}
    \begin{split}
        [(A + E_{1})q_{1}, (A + E_{2})q_{2},\dots, (A + E_{k})q_{k}] &= Q_{k+1}H_{k}\\
        (A + \mathcal{E}_{k})Q_{k} &= Q_{k+1}H_{k}, \hspace{0.1in} \mathcal{E}_{k} = \sum_{i=1}^{k}E_{i}q_{i}q_{i}^{t}\\
        \mathcal{A}Q_{k} &= W_{k}
    \end{split}
\end{align}
Where $W_{m} = Q_{m+1}H_{m}$ from this point foward.

Now the subspace spawn by the vectors of $Q_{k}$ is not the Krylov's subspace $\mathcal{K}_{k}(A,b)$, but these are still orthonormal.  To see what kind of subspace our new $Q$ spams, \ref{eq:new_projection} is looked into in \ref{eq:q_basis}.

\begin{align}\label{eq:q_basis}
    \begin{split}
        (A + E_{k})q_{k}=\mathcal{A}_{k} q_{k} = h_{1,k}q_{1} + h_{2,k}q_{2} + \dots + h_{k+1,k}q_{k+1}
    \end{split}
\end{align}

For $k=1$, we have that $q_{2}$ is a combination of the vectors $\mathcal{A}_{1}b$ and $b$ (since $q_{1}$ = b). For $k=2$ we see that $q_{3}$ is a combination that involves $\mathcal{A}_{2} \mathcal{A}_{1}b$ and so forth.

Expression \ref{eq:new_projection} then shows that $Q_{k}$ becomes a basis for a new Krylov's subspace, $\mathcal{K}_{k}(A+\mathcal{E}_{k},b)$ $= \spn \{b,\mathcal{A}_{1}b,\dots, \mathcal{A}_{k}\dots\mathcal{A}_{1}b \}$, made by a large pertubation in $A$, that gets updated in each iteration.

A new distinction should also be made between the two types of residuals appearing in the process: $r_{k}$, the exact residual of an iteration, and $\tilde{r}_{k}$, the one that will really be calculated. A detailed definition for both and a measure of how distant they are is in \ref{eq:res_relation}.

\begin{align}\label{eq:res_relation}
    \begin{split}
        r_{k} &= r_{0} - AQ_{k}y_{k}\\
        &= r_{0} - (Q_{k+1}H_{k} - [E_{1}q_{1},\dots , E_{k}q_{k}])y_{k}\\
        &= \tilde{r}_{k} +[E_{1}q_{1},\dots , E_{k}q_{k}]y_{k}\\
        \rightarrow \delta_{k} &= \norm{r_{k} - \tilde{r}_{k}}  = \norm{[E_{1}q_{1},\dots , E_{k}q_{k}]y_{k}}
    \end{split}
\end{align}

Considering $y_{k} = [\eta_{1}^{(k)} \dots \eta_{n}^{(k)} ] $, upper index to clarify the iteration, an upper bound for $\delta_{k}$ can be found, but before we go through \ref{eq:res_rightside}.

\begin{align}\label{eq:res_rightside}
    \begin{split}
        \norm{[E_{1}q_{1},\dots , E_{k}q_{k}]y_{k}} & = \norm{\sum_{i=1}^{k}E_{i}q_{i}\eta^{(k)}_{i}}\\
        \norm{\sum_{i=1}^{k}E_{i}q_{i}\eta^{(k)}_{i}} & \leq \sum_{i=1}^{k}\norm{E_{i}}\norm{q_{i}\eta^{(k)}_{i}}\\
        \norm{\sum_{i=1}^{k}E_{i}q_{i}\eta^{(k)}_{i}} & \leq \sum_{i=1}^{k}\norm{E_{i}}|\eta^{(k)}_{i}|
    \end{split}
\end{align}

We use the fact that $q_{i}$ are unitary between the last two lines. The bound on $\delta_{k}$ is then found in \ref{eq:borne_delta}.


\begin{equation}\label{eq:borne_delta}
    \delta_{k} = \norm{r_{k} - \tilde{r}_{k}} \leq \sum_{i=1}^{k} \norm{E_{i}} \norm{\eta_{i}^{(k)}}
\end{equation}

\ref{eq:borne_delta} tells us that in order to keep both residuals close, either the pertubation of $A$, somewhat measured by $\norm{E_{i}}$, or the elements of $y_{i}$ should be kept small. Since we expect to use more \textit{relaxed} approximations of $A$ as the iterations go on, a greater tolerance in $E_{k}$ could be compensated with a sufficiently small $y_{k}$.


The problem is $y_{k}$ is only found after the construction of $E_{k}$, so an upper bound must be also found for its value.

Knowing $y_{k}$ is the solution of the minimisation of $ \norm{H_{k}y_{k} - e_{1}\beta} $, we consider $\Omega_{k} =G(k,k+1) G(k-1,k) \dots G(1,2)$ where each $G$ represents a Givens rotation as shown in \ref{eq:givens}, so $\Omega_{k}$ is the matrix that transforms $H_{k}$ into an upper triangular matrix.

The aplication of $\Omega_{k}$ in either side of $H_{k}y_{k}  = e_{1}\beta$ gives us \ref{eq:y_limit_start}.

\begin{align}\label{eq:y_limit_start}
    \begin{split}
        \Omega_{k}H_{k}y_{k}&=\Omega_{k}e_{1}\beta\\
        R_{k}y_{k}&=g_{k}\\
        y_{k}&=R^{-1}_{k}g_{k}
    \end{split}
\end{align}

Since $R_{k}$, the transformation of a Hessenberg matrix by a series of Givens rotations, is upper triangular, then its inverse also is.
Being an upper triangular matrix, the first $i-1$ elements of its ith line are zeros, so using Matlab index notation in \ref{eq:R_elements}.

\begin{equation}\label{eq:R_elements}
    (R^{-1}_{k})_{i,1:k}(g_{k})_{1:k} = (R^{-1}_{k})_{i,i:k}(g_{k})_{i:k}
\end{equation}

Using this last result in \ref{eq:y_limit_start} gives \ref{eq:y_limit}.

\begin{align}\label{eq:y_limit}
    \begin{split}
        |\eta^{(k)}_{i}| &= \norm{(R^{-1}_{k})_{i,i:k}(g_{k})_{i:k}}\\
        |\eta^{(k)}_{i}| &\leq \norm{e_{k}R^{-1}_{k}} \norm{(g_{k})_{i:k}}\\
        |\eta^{(k)}_{i}| &\leq \norm{e_{k}R^{-1}_{k}} \norm{(g_{k})_{i:k}}\\
    \end{split}
\end{align}

Since $\norm{e_{k}R^{-1}_{k}} \leq \norm{R^{-1}_{k}} = \sigma_{k}(H_{k})^{-1}$ and $\norm{(g_{k})_{i:k}} \leq \norm{\tilde{r}_{i-1}}$ \cite{simoncini2003theory}, the bound is given by \ref{eq:bound_yinexact}.

\begin{equation}\label{eq:bound_yinexact}
    \norm{\eta_{i}^{(k)}} \leq \frac{1}{\sigma_{k}(H_{k})} \norm{\tilde{r}_{i-1}}
\end{equation}

Putting \ref{eq:bound_yinexact} in \ref{eq:borne_delta} gives the results in \ref{eq:boundE_intermediate}. Setting $\delta_{k} \leq \epsilon$ and determining a bound for each $\norm{E_{i}}$ gets us \ref{eq:bound_E}.


\begin{equation}\label{eq:boundE_intermediate}
    \delta_{k} \leq \sum_{i=1}^{k} \frac{\norm{E_{i}}}{\sigma_{k}(H_{k})}\norm{\tilde{r}_{i-1}}
\end{equation}

\begin{equation}\label{eq:bound_E}
    \norm{E_{i}} \leq \frac{\sigma_{k}(H_{k})\epsilon}{k\norm{\tilde{r}_{i-1}}}
\end{equation}

Since $H_{k}$ is also one of the matrices being constructed throughout the method, a workaround is necessary to apply find these bounds in a pratical situation. Either using an estimation of $\sigma_{k}(H_{k})$ with the singular values of $A$ or grouping all uncalculated terms in an $\ell_{k}$ that will be estimated empirically \cite{simoncini2003theory}, obtaining \ref{eq:boundE_final}.

\begin{equation}\label{eq:boundE_final}
    \norm{E_{i}} \leq \ell_{k} \frac{1}{\norm{\tilde{r}_{i-1}}} \epsilon
\end{equation}

It should be noted \cite{simoncini2003theory} that among te initial bounds, some aren't really sharp, mainly \ref{eq:res_rightside} and \ref{eq:bound_yinexact}, and further empirical analysis of these bounds could show a better theoretical bound can be found for both. A plot of these bounds for the cavity problem that will be studied is shown later.

It should also be noted that $\tilde{r}_{k}$, after the transformation of $H_{k}$ into an upper triangular matrix, is also found in the $i+1$'th element of $g_{m}$ in \ref{eq:y_limit_start}. The demonstration follow the same proof as for $g_{m}$ in \ref{eq:after_givens}, given that $y_{m}$ is a solution to a linear system that involves an upper triangular matrix.

The remaining theory in this report also explains the basics of Hierarchical Matrices, the structure that will be used to compress the matrices used in the algorithm, since $A$ appears in the discretization of integral operators and uses large dimensions.
It's also though these strucures each $E_{k}$ will be made. As it will be explained later, at each iteration an $E_{k}$ will be indirectly constructed during the inexact product $\mathcal{A}_{k}q$, mainly using the residues that appear during this structure's construction, in the iterations of the $ACA Method$.